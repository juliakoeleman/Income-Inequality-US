% Options for packages loaded elsewhere
\PassOptionsToPackage{unicode}{hyperref}
\PassOptionsToPackage{hyphens}{url}
%
\documentclass[
]{article}
\usepackage{amsmath,amssymb}
\usepackage{iftex}
\ifPDFTeX
  \usepackage[T1]{fontenc}
  \usepackage[utf8]{inputenc}
  \usepackage{textcomp} % provide euro and other symbols
\else % if luatex or xetex
  \usepackage{unicode-math} % this also loads fontspec
  \defaultfontfeatures{Scale=MatchLowercase}
  \defaultfontfeatures[\rmfamily]{Ligatures=TeX,Scale=1}
\fi
\usepackage{lmodern}
\ifPDFTeX\else
  % xetex/luatex font selection
\fi
% Use upquote if available, for straight quotes in verbatim environments
\IfFileExists{upquote.sty}{\usepackage{upquote}}{}
\IfFileExists{microtype.sty}{% use microtype if available
  \usepackage[]{microtype}
  \UseMicrotypeSet[protrusion]{basicmath} % disable protrusion for tt fonts
}{}
\makeatletter
\@ifundefined{KOMAClassName}{% if non-KOMA class
  \IfFileExists{parskip.sty}{%
    \usepackage{parskip}
  }{% else
    \setlength{\parindent}{0pt}
    \setlength{\parskip}{6pt plus 2pt minus 1pt}}
}{% if KOMA class
  \KOMAoptions{parskip=half}}
\makeatother
\usepackage{xcolor}
\usepackage[margin=1in]{geometry}
\usepackage{color}
\usepackage{fancyvrb}
\newcommand{\VerbBar}{|}
\newcommand{\VERB}{\Verb[commandchars=\\\{\}]}
\DefineVerbatimEnvironment{Highlighting}{Verbatim}{commandchars=\\\{\}}
% Add ',fontsize=\small' for more characters per line
\usepackage{framed}
\definecolor{shadecolor}{RGB}{248,248,248}
\newenvironment{Shaded}{\begin{snugshade}}{\end{snugshade}}
\newcommand{\AlertTok}[1]{\textcolor[rgb]{0.94,0.16,0.16}{#1}}
\newcommand{\AnnotationTok}[1]{\textcolor[rgb]{0.56,0.35,0.01}{\textbf{\textit{#1}}}}
\newcommand{\AttributeTok}[1]{\textcolor[rgb]{0.13,0.29,0.53}{#1}}
\newcommand{\BaseNTok}[1]{\textcolor[rgb]{0.00,0.00,0.81}{#1}}
\newcommand{\BuiltInTok}[1]{#1}
\newcommand{\CharTok}[1]{\textcolor[rgb]{0.31,0.60,0.02}{#1}}
\newcommand{\CommentTok}[1]{\textcolor[rgb]{0.56,0.35,0.01}{\textit{#1}}}
\newcommand{\CommentVarTok}[1]{\textcolor[rgb]{0.56,0.35,0.01}{\textbf{\textit{#1}}}}
\newcommand{\ConstantTok}[1]{\textcolor[rgb]{0.56,0.35,0.01}{#1}}
\newcommand{\ControlFlowTok}[1]{\textcolor[rgb]{0.13,0.29,0.53}{\textbf{#1}}}
\newcommand{\DataTypeTok}[1]{\textcolor[rgb]{0.13,0.29,0.53}{#1}}
\newcommand{\DecValTok}[1]{\textcolor[rgb]{0.00,0.00,0.81}{#1}}
\newcommand{\DocumentationTok}[1]{\textcolor[rgb]{0.56,0.35,0.01}{\textbf{\textit{#1}}}}
\newcommand{\ErrorTok}[1]{\textcolor[rgb]{0.64,0.00,0.00}{\textbf{#1}}}
\newcommand{\ExtensionTok}[1]{#1}
\newcommand{\FloatTok}[1]{\textcolor[rgb]{0.00,0.00,0.81}{#1}}
\newcommand{\FunctionTok}[1]{\textcolor[rgb]{0.13,0.29,0.53}{\textbf{#1}}}
\newcommand{\ImportTok}[1]{#1}
\newcommand{\InformationTok}[1]{\textcolor[rgb]{0.56,0.35,0.01}{\textbf{\textit{#1}}}}
\newcommand{\KeywordTok}[1]{\textcolor[rgb]{0.13,0.29,0.53}{\textbf{#1}}}
\newcommand{\NormalTok}[1]{#1}
\newcommand{\OperatorTok}[1]{\textcolor[rgb]{0.81,0.36,0.00}{\textbf{#1}}}
\newcommand{\OtherTok}[1]{\textcolor[rgb]{0.56,0.35,0.01}{#1}}
\newcommand{\PreprocessorTok}[1]{\textcolor[rgb]{0.56,0.35,0.01}{\textit{#1}}}
\newcommand{\RegionMarkerTok}[1]{#1}
\newcommand{\SpecialCharTok}[1]{\textcolor[rgb]{0.81,0.36,0.00}{\textbf{#1}}}
\newcommand{\SpecialStringTok}[1]{\textcolor[rgb]{0.31,0.60,0.02}{#1}}
\newcommand{\StringTok}[1]{\textcolor[rgb]{0.31,0.60,0.02}{#1}}
\newcommand{\VariableTok}[1]{\textcolor[rgb]{0.00,0.00,0.00}{#1}}
\newcommand{\VerbatimStringTok}[1]{\textcolor[rgb]{0.31,0.60,0.02}{#1}}
\newcommand{\WarningTok}[1]{\textcolor[rgb]{0.56,0.35,0.01}{\textbf{\textit{#1}}}}
\usepackage{graphicx}
\makeatletter
\newsavebox\pandoc@box
\newcommand*\pandocbounded[1]{% scales image to fit in text height/width
  \sbox\pandoc@box{#1}%
  \Gscale@div\@tempa{\textheight}{\dimexpr\ht\pandoc@box+\dp\pandoc@box\relax}%
  \Gscale@div\@tempb{\linewidth}{\wd\pandoc@box}%
  \ifdim\@tempb\p@<\@tempa\p@\let\@tempa\@tempb\fi% select the smaller of both
  \ifdim\@tempa\p@<\p@\scalebox{\@tempa}{\usebox\pandoc@box}%
  \else\usebox{\pandoc@box}%
  \fi%
}
% Set default figure placement to htbp
\def\fps@figure{htbp}
\makeatother
\setlength{\emergencystretch}{3em} % prevent overfull lines
\providecommand{\tightlist}{%
  \setlength{\itemsep}{0pt}\setlength{\parskip}{0pt}}
\setcounter{secnumdepth}{-\maxdimen} % remove section numbering
\usepackage{bookmark}
\IfFileExists{xurl.sty}{\usepackage{xurl}}{} % add URL line breaks if available
\urlstyle{same}
\hypersetup{
  pdftitle={The Effect of Education Level on Income Inequality in the United States (1990--2000)},
  pdfauthor={Carlijn Calori (2860228), Leah Delikát (2813155), Fadhil Dhafir (2847129),; Sten Groen (2813781), Julia Koeleman (2839542), Anne Schrama (2812834),; Marie-Louise Stevens (2814083), Sophia Zentgraf (2853242)},
  hidelinks,
  pdfcreator={LaTeX via pandoc}}

\title{The Effect of Education Level on Income Inequality in the United
States (1990--2000)}
\author{Carlijn Calori (2860228), Leah Delikát (2813155), Fadhil Dhafir
(2847129), \and Sten Groen (2813781), Julia Koeleman (2839542), Anne
Schrama (2812834), \and Marie-Louise Stevens (2814083), Sophia Zentgraf
(2853242)}
\date{2025-06-24}

\begin{document}
\maketitle

\section{Title Page}\label{title-page}

Carlijn Calori, Leah Delikát, Fadhil Dhafir, Sten Groen, Julia Koeleman,
Anne Schrama, Marie-Louise Stevens \& Sophia Zentgraf

Tutorial: Group 1

Tutor: J.F. Fitzgerald

\section{1 Identification of the Social
Problem}\label{identification-of-the-social-problem}

\subsection{1.1 Describe the Social
Problem}\label{describe-the-social-problem}

Income inequality is a serious social issue in the United States. Over
the past few decades, the gap between the highest and lowest earners has
grown a lot, leading to problems on social, economic, and political
levels. This widening gap creates unequal opportunities, adds pressure
to communities, and makes it harder for people to succeed in the job
market. Education plays an important role in this situation. People with
higher levels of education tend to earn more money than those with lower
levels of education. But how straightforward is that relationship?

Several sources have pointed out that income inequality is a major
concern. The Organisation for Economic Co-operation and Development
(OECD) has stated in multiple reports that the U.S. has one of the
highest levels of income inequality among developed countries (Denk et
al., 2013). They point out that this can slow down economic growth and
limit social mobility. Horowitz et al.~(2020) from the Pew Research
Center has also found that income gaps keep growing and that education
is a key factor---people without higher education are falling further
behind in the labor market. Basu and Stiglitz (2014) point out that
individuals in the top income decile tend to have relatively high levels
of education, whereas those in the bottom four deciles often have
limited or poor-quality schooling. Denk et al.~(2013) and Horowitz et
al.~(2020) show that income inequality, and the role education plays in
it, is a long-term issue that needs real attention.

Even tough this topic has been researched before, this analysis takes a
closer look than many past studies. We focus on differences between U.S.
states to see how the link between education and income changes from one
region to another. By doing this, we can offer a more detailed and
complete picture of income inequality. With this more in-depth analysis,
we hope to find new insights that haven't been explored in other
research.

\section{2 Data Sourcing}\label{data-sourcing}

\subsection{2.1 Description and
limitations}\label{description-and-limitations}

The dataset ``Inequality and Growth in the United States: Evidence from
a New State-Level Panel of Income Inequality Measure'' of the researcher
Frank has been used for our project. This dataset provides historical
coverage (1990-2000) of the highest income group in the US. Another
reason why this data set has been used is because it specifically shows
the top income groups per state. This way, the differences per state can
be easily compared to one another. The dataset shows multiple variables
that can measure inequality. The data is not from a government website,
but from a researcher himself. However, the researcher did use data from
the World Inequality Database to calculate the different top income
groups. The limitation of this dataset is that it only provides the top
income percentages, so there is not a possibility to compare the bottom
10 percent with the top 10, which makes it harder to accurately measure
inequality.

The second dataset is ``The Educational attainment of persons 25 years
old and over, by race/ethnicity and state: April 1990 and April 2000''.
This dataset focuses on education level and has been retrieved from the
National Center form Education Statistics (NCES). We chose this dataset
because the level of education is a key indicator of inequality. The
higher the education, the higher the income. Therefore big differences
between education level will lead to a greater inequality. Furthermore,
the dataset specifies the education level across all U.S. states. Which
we can combine with the dataset of the top income percentages.
Differences across racial and ethnic groups are also shown in the data,
but will not be considered here, as they fall outside the scope of this
analysis. The limitation of the dataset is that it includes data about
only two points in time: April 1990 and April 2000. Unfortunately, there
is no data available for the years in between.

\subsection{2.2 Load in the data}\label{load-in-the-data}

To load the datasets, a link is provided to a Google Drive folder, from
which a ZIP file containing the two datasets can be downloaded.

\begin{Shaded}
\begin{Highlighting}[]
\CommentTok{\#Load the files}
\NormalTok{Distribution }\OtherTok{\textless{}{-}} \FunctionTok{read\_excel}\NormalTok{(}\StringTok{"datasets\_income\_inequality/Frank\_WID\_2020.xls"}\NormalTok{, }\AttributeTok{sheet =} \DecValTok{3}\NormalTok{)}
\NormalTok{Education }\OtherTok{\textless{}{-}} \FunctionTok{read\_excel}\NormalTok{(}\StringTok{"datasets\_income\_inequality/tabn012.xls"}\NormalTok{, }\AttributeTok{col\_types =} \FunctionTok{c}\NormalTok{(}\StringTok{"text"}\NormalTok{, }\StringTok{"skip"}\NormalTok{, }\StringTok{"text"}\NormalTok{, }\StringTok{"skip"}\NormalTok{, }\StringTok{"skip"}\NormalTok{, }\StringTok{"skip"}\NormalTok{, }\StringTok{"skip"}\NormalTok{, }\StringTok{"skip"}\NormalTok{, }\StringTok{"skip"}\NormalTok{, }\StringTok{"skip"}\NormalTok{, }\StringTok{"skip"}\NormalTok{, }\StringTok{"skip"}\NormalTok{, }\StringTok{"skip"}\NormalTok{, }\StringTok{"skip"}\NormalTok{, }\StringTok{"text"}\NormalTok{, }\StringTok{"skip"}\NormalTok{, }\StringTok{"skip"}\NormalTok{, }\StringTok{"skip"}\NormalTok{, }\StringTok{"skip"}\NormalTok{, }\StringTok{"skip"}\NormalTok{, }\StringTok{"skip"}\NormalTok{, }\StringTok{"skip"}\NormalTok{, }\StringTok{"skip"}\NormalTok{, }\StringTok{"skip"}\NormalTok{, }\StringTok{"skip"}\NormalTok{, }\StringTok{"skip"}\NormalTok{, }\StringTok{"text"}\NormalTok{, }\StringTok{"skip"}\NormalTok{, }\StringTok{"skip"}\NormalTok{, }\StringTok{"skip"}\NormalTok{, }\StringTok{"skip"}\NormalTok{, }\StringTok{"skip"}\NormalTok{, }\StringTok{"skip"}\NormalTok{,}\StringTok{"skip"}\NormalTok{, }\StringTok{"skip"}\NormalTok{, }\StringTok{"skip"}\NormalTok{, }\StringTok{"skip"}\NormalTok{, }\StringTok{"skip"}\NormalTok{, }\StringTok{"text"}\NormalTok{, }\StringTok{"skip"}\NormalTok{, }\StringTok{"skip"}\NormalTok{, }\StringTok{"skip"}\NormalTok{, }\StringTok{"skip"}\NormalTok{, }\StringTok{"skip"}\NormalTok{, }\StringTok{"skip"}\NormalTok{, }\StringTok{"skip"}\NormalTok{, }\StringTok{"skip"}\NormalTok{, }\StringTok{"skip"}\NormalTok{, }\StringTok{"skip"}\NormalTok{))}
\end{Highlighting}
\end{Shaded}

\begin{verbatim}
## New names:
## * `` -> `...2`
## * `` -> `...3`
## * `` -> `...4`
## * `` -> `...5`
\end{verbatim}

\section{3 Quantifying}\label{quantifying}

\subsection{3.1 Data Cleaning}\label{data-cleaning}

Only the years 1990 to 2000 are kept in the Distribution dataset. This
is because the Education dataset includes data only for 1990 and 2000.
To ensure the datasets can be merged and compared properly, just 1990,
2000, and the years in between are included.

In addition, two rows are removed, as they are not a U.S. state. One of
the rows corresponds to the entire United States, while the other
represents the District of Columbia.

\begin{Shaded}
\begin{Highlighting}[]
\NormalTok{Distribution }\OtherTok{\textless{}{-}}\NormalTok{ Distribution }\SpecialCharTok{\%\textgreater{}\%}
  \FunctionTok{filter}\NormalTok{(Year }\SpecialCharTok{\%in\%} \DecValTok{1990}\SpecialCharTok{:}\DecValTok{2000}\NormalTok{, }
\NormalTok{         State }\SpecialCharTok{!=} \FunctionTok{c}\NormalTok{(}\StringTok{"United States"}\NormalTok{, }\StringTok{"District of Columbia"}\NormalTok{))}
\end{Highlighting}
\end{Shaded}

For the Education dataset, unnecessary rows are removed. These rows
contain no data and consist only of blank spaces between entries. Also,
two rows representing the entire United States and the District of
Columbia are removed.

To simplify the datatset, columns are renamed to reflect the type of
data they contain, and any dots following state names are removed.

For the final step of cleaning, before merging the two datasets, the
Education dataset needs to be converted to a long format. Currently, it
contains four separate variables for the years 1990 and 2000, but these
need to be combined into a single column with the years listed
vertically. This transformation allows the dataset to be merged properly
with the Distribution dataset, which is already in long format.

\begin{Shaded}
\begin{Highlighting}[]
\NormalTok{Education }\OtherTok{\textless{}{-}}\NormalTok{ Education[ }\SpecialCharTok{{-}}\FunctionTok{c}\NormalTok{(}\DecValTok{1}\SpecialCharTok{:}\DecValTok{13}\NormalTok{, }\DecValTok{14}\NormalTok{, }\DecValTok{20}\NormalTok{, }\DecValTok{24}\NormalTok{, }\DecValTok{26}\NormalTok{, }\DecValTok{32}\NormalTok{, }\DecValTok{38}\NormalTok{, }\DecValTok{44}\NormalTok{, }\DecValTok{50}\NormalTok{, }\DecValTok{56}\NormalTok{, }\DecValTok{62}\NormalTok{, }\DecValTok{68}\NormalTok{, }\DecValTok{75}\SpecialCharTok{:}\DecValTok{79}\NormalTok{), ]}
\end{Highlighting}
\end{Shaded}

\begin{Shaded}
\begin{Highlighting}[]
\FunctionTok{names}\NormalTok{(Education)[}\DecValTok{1}\SpecialCharTok{:}\DecValTok{5}\NormalTok{] }\OtherTok{\textless{}{-}} \FunctionTok{c}\NormalTok{(}\StringTok{"State"}\NormalTok{, }\StringTok{"Highschool1990"}\NormalTok{, }\StringTok{"Highschool2000"}\NormalTok{, }\StringTok{"Bachelor1990"}\NormalTok{, }\StringTok{"Bachelor2000"}\NormalTok{)}
\end{Highlighting}
\end{Shaded}

\begin{Shaded}
\begin{Highlighting}[]
\NormalTok{Education }\OtherTok{\textless{}{-}}\NormalTok{ Education }\SpecialCharTok{\%\textgreater{}\%} 
  \FunctionTok{mutate}\NormalTok{(}\AttributeTok{State =}\NormalTok{ State }\SpecialCharTok{\%\textgreater{}\%}
           \FunctionTok{str\_replace\_all}\NormalTok{(}\StringTok{"…+"}\NormalTok{, }\StringTok{""}\NormalTok{) }\SpecialCharTok{\%\textgreater{}\%}
           \FunctionTok{str\_replace\_all}\NormalTok{(}\StringTok{"}\SpecialCharTok{\textbackslash{}\textbackslash{}}\StringTok{.+"}\NormalTok{, }\StringTok{""}\NormalTok{) }\SpecialCharTok{\%\textgreater{}\%}
           \FunctionTok{str\_trim}\NormalTok{()) }\CommentTok{\# to delete all the dots after the state }
\end{Highlighting}
\end{Shaded}

\begin{Shaded}
\begin{Highlighting}[]
\NormalTok{df1990 }\OtherTok{\textless{}{-}} \FunctionTok{data.frame}\NormalTok{(}\AttributeTok{State =}\NormalTok{ Education}\SpecialCharTok{$}\NormalTok{State, }\AttributeTok{Year =} \DecValTok{1990}\NormalTok{, }\AttributeTok{Highschool =}\NormalTok{ Education}\SpecialCharTok{$}\NormalTok{Highschool1990, }\AttributeTok{Bachelor =}\NormalTok{ Education}\SpecialCharTok{$}\NormalTok{Bachelor1990, }\AttributeTok{stringsAsFactors =} \ConstantTok{FALSE}\NormalTok{)}

\NormalTok{df2000 }\OtherTok{\textless{}{-}} \FunctionTok{data.frame}\NormalTok{(}\AttributeTok{State =}\NormalTok{ Education}\SpecialCharTok{$}\NormalTok{State,}\AttributeTok{Year =} \DecValTok{2000}\NormalTok{,}\AttributeTok{Highschool =}\NormalTok{ Education}\SpecialCharTok{$}\NormalTok{Highschool2000,}\AttributeTok{Bachelor =}\NormalTok{ Education}\SpecialCharTok{$}\NormalTok{Bachelor2000,}\AttributeTok{stringsAsFactors =} \ConstantTok{FALSE}\NormalTok{)}

\NormalTok{Education }\OtherTok{\textless{}{-}} \FunctionTok{rbind}\NormalTok{(df1990, df2000)}

\NormalTok{Education }\OtherTok{\textless{}{-}}\NormalTok{ Education }\SpecialCharTok{\%\textgreater{}\%}
  \FunctionTok{mutate}\NormalTok{(}\FunctionTok{across}\NormalTok{(}\DecValTok{3}\SpecialCharTok{:}\DecValTok{4}\NormalTok{, }\SpecialCharTok{\textasciitilde{}} \FunctionTok{as.numeric}\NormalTok{(.)))}
\end{Highlighting}
\end{Shaded}

To merge the datasets, the full\_join() function is applied. This
function is appropriate because it performs an outer join, retaining all
observations from both datasets based on the specific join keys (e.g.,
State and Year). Since the distribution dataset includes data for each
year from 1990 to 2000, and the Education dataset only includes data for
1990 and 2000, full\_join() ensures that no rows are lost during the
merge. Missing values are introduced where data is not available in one
of the datasets, allowing for a complete and alinged structure for
subsequent analysis. After the merge, all numeric values are rounded to
two decimals.

\begin{Shaded}
\begin{Highlighting}[]
\NormalTok{Inequality }\OtherTok{\textless{}{-}}\NormalTok{ Distribution }\SpecialCharTok{\%\textgreater{}\%}
  \FunctionTok{full\_join}\NormalTok{(Education, }\AttributeTok{by =} \FunctionTok{c}\NormalTok{(}\StringTok{"State"}\NormalTok{, }\StringTok{"Year"}\NormalTok{))}

\NormalTok{Inequality }\OtherTok{\textless{}{-}}\NormalTok{ Inequality }\SpecialCharTok{\%\textgreater{}\%}
  \FunctionTok{select}\NormalTok{(}\SpecialCharTok{{-}}\NormalTok{st) }\SpecialCharTok{\%\textgreater{}\%}
  \FunctionTok{mutate}\NormalTok{(}\FunctionTok{across}\NormalTok{(}\FunctionTok{where}\NormalTok{(is.numeric), }\SpecialCharTok{\textasciitilde{}} \FunctionTok{round}\NormalTok{(.x, }\DecValTok{2}\NormalTok{)))}
\end{Highlighting}
\end{Shaded}

\subsection{3.2 Describing the type of variables in the
datasets}\label{describing-the-type-of-variables-in-the-datasets}

\begin{Shaded}
\begin{Highlighting}[]
\FunctionTok{head}\NormalTok{(Inequality)}
\end{Highlighting}
\end{Shaded}

\begin{verbatim}
## # A tibble: 6 x 10
##    Year State      Top10_adj Top5_adj Top1_adj Top05_adj Top01_adj Top001_adj
##   <dbl> <chr>          <dbl>    <dbl>    <dbl>     <dbl>     <dbl>      <dbl>
## 1  1990 Alabama         39.9     26.0     12.4      9.1       4.74       1.78
## 2  1990 Alaska          34.0     22.7     11.2      8.74      4.93       2.15
## 3  1990 Arizona         38.9     26.5     12.3      9.25      4.56       1.64
## 4  1990 Arkansas        38.2     25.6     12.4      8.84      4.72       1.8 
## 5  1990 California      42.8     30.6     16.6     12.5       6.82       2.84
## 6  1990 Colorado        36.4     25.4     12.0      9.16      4.74       1.82
## # i 2 more variables: Highschool <dbl>, Bachelor <dbl>
\end{verbatim}

This is the merged dataset, named ``Inequality''. The first variable
indicates the year (i.e., from 1990 to 2000), and the second column
identifies the state from which the data originates. Each state appears
eleven times in the dataset - once for every year.

The next five variables represent the percentage of total income earned
by specific income groups. For example, the first of these columns shows
that the top 10 percent of earners in Alabama in 1990 earned 39.9\% of
the total income. The following four columns follow the same structure,
with each representing increasingly smaller income groups.

The final two columns capture educational attainment levels, expressed
in percentages. For instance, in 1990, 66.9\% of the population in
Alabama had a high school diploma or higher and 15.7\% held a bachelor's
degree or higher.

\subsection{3.2 Creating new variables}\label{creating-new-variables}

To measure income inequality, a new variable is constructed, modeled
after the Palma Ratio. This ratio is defined as ``the ratio of the
income share of the top 10 percent over that of the bottom 40 percent''
(Basu \& Stiglitz, 2016, p.~xxvi), and is designed to capture inequality
at the extremes of the income distribution. However, the Distribution
dataset does not include the income share of the bottom 40 percent.
Therefore, as a practical and intuitive alternative, inequality is
measured by dividing the income share of the top 10 percent by that of
the bottom 90 percent (i.e., 100\% minus the top 10\%).

Using this measure, inequality will be visualized over time by
presenting the mean and median values of all U.S. states. This
visualization provides a broad overview of the national trend in
inequality across the country, showing whether inequality is generally
increasing or decreasing over time.

Additionally, a second new variable groups states into four categories -
Low, Medium Low, Medium High, and High - based on the percentage of the
population with a bachelor's degree or higher, allowing inequality to be
examined across sub-populations. This approach helps reveal whether
states with a higher share of highly educated residents tend to have a
higher or lower inequality.

Finally, inequality will be visualized across the 50 states through a
third new variable that shows the absolute change in inequality between
1990 and 2000. Not all states experience inequality changes equally;
some show significant increases while others remain stable. This
variation can be attributed to factors such as local institutions, tax
policies, or demographic shifts (Wei, 2015; Arcabic et al., 2021).

\begin{Shaded}
\begin{Highlighting}[]
\NormalTok{Inequality }\OtherTok{\textless{}{-}}\NormalTok{ Inequality }\SpecialCharTok{\%\textgreater{}\%}
  \FunctionTok{mutate}\NormalTok{(}\AttributeTok{Top10\_vs\_bottom90 =} \FunctionTok{round}\NormalTok{(Top10\_adj}\SpecialCharTok{/}\NormalTok{(}\DecValTok{100}\SpecialCharTok{{-}}\NormalTok{Top10\_adj), }\DecValTok{2}\NormalTok{))}
\end{Highlighting}
\end{Shaded}

\begin{Shaded}
\begin{Highlighting}[]
\NormalTok{Inequality }\OtherTok{\textless{}{-}}\NormalTok{ Inequality }\SpecialCharTok{\%\textgreater{}\%}
  \FunctionTok{group\_by}\NormalTok{(Year) }\SpecialCharTok{\%\textgreater{}\%}
  \FunctionTok{mutate}\NormalTok{(}\AttributeTok{Quartile =} \FunctionTok{ntile}\NormalTok{(Bachelor, }\DecValTok{4}\NormalTok{), }\AttributeTok{Bachelor\_level =} \FunctionTok{case\_when}\NormalTok{(}
\NormalTok{    Quartile }\SpecialCharTok{==} \DecValTok{1} \SpecialCharTok{\textasciitilde{}} \StringTok{"Low"}\NormalTok{, }
\NormalTok{    Quartile }\SpecialCharTok{==} \DecValTok{2} \SpecialCharTok{\textasciitilde{}} \StringTok{"Medium Low"}\NormalTok{, }
\NormalTok{    Quartile }\SpecialCharTok{==} \DecValTok{3} \SpecialCharTok{\textasciitilde{}} \StringTok{"Medium High"}\NormalTok{,}
\NormalTok{    Quartile }\SpecialCharTok{==} \DecValTok{4} \SpecialCharTok{\textasciitilde{}} \StringTok{"High"}\NormalTok{))}

\CommentTok{\# Make sure Education is a factor with the right order}
\NormalTok{Inequality}\SpecialCharTok{$}\NormalTok{Bachelor\_level }\OtherTok{\textless{}{-}} \FunctionTok{factor}\NormalTok{(}
\NormalTok{  Inequality}\SpecialCharTok{$}\NormalTok{Bachelor\_level,}
  \AttributeTok{levels =} \FunctionTok{c}\NormalTok{(}\StringTok{"Low"}\NormalTok{, }\StringTok{"Medium Low"}\NormalTok{, }\StringTok{"Medium High"}\NormalTok{, }\StringTok{"High"}\NormalTok{))}
\end{Highlighting}
\end{Shaded}

\subsection{3.3 Temporal Variation}\label{temporal-variation}

\pandocbounded{\includegraphics[keepaspectratio]{Income_Inequality_files/figure-latex/unnamed-chunk-14-1.pdf}}

The graph illustrates the development of income inequality in the United
States from 1990 to 2000, using the ratio between the income share of
the top 10 percent and the bottom 90 percent. Both the median (blue) and
mean (red) values show a clear upward trend over time, with a notable
acceleration after 1994. The increasing gap between mean and median also
suggest that the highest earners experienced disproportionate income
growth compared to the general population.

\subsection{3.4 Sub-population
Variation}\label{sub-population-variation}

\pandocbounded{\includegraphics[keepaspectratio]{Income_Inequality_files/figure-latex/unnamed-chunk-15-1.pdf}}

The box-and-whisker plot displays income inequality in the year 2000
across four groups of states, categorized by the percentage of the
population with a bachelor's degree or higher. The results show a
positive correlation: states with a higher level of educational
attainment tend to exhibit greater income inequality. This suggests that
while higher education leads to better-paying jobs, it also contributes
to a wider income gap, as the earnings of highly educated individuals
pull away from those with lower education levels.

\subsection{3.5 Spatial Variation}\label{spatial-variation}

\pandocbounded{\includegraphics[keepaspectratio]{Income_Inequality_files/figure-latex/unnamed-chunk-18-1.pdf}}

The spatial variation graph shows the change in income inequality
between 1990 and 2000 across all U.S. states, measured by the
Top10\_vs\_bottom90 ratio. Most states experienced an increase in
inequality, though the magnitude of change varies. The most significant
increases are observed in states such as Massachusetts, California, and
Arizona, indicating that regional economic developments may have
amplified income disparities. Conversely, several Midwestern and
Southern states show smaller changes, and a few appear relatively stable
or even slightly declining.

This map effectively highlights regional disparities, allowing for
geographic patterns in inequality growth to become visible. However, the
use of a continuous color gradient lacks categorical thresholds, making
it harder to distinguish what constitutes a mild versus a substantial
increase in inequality. For example, grouping values into low, medium,
and high changes could make regional patterns more interpretable.
Additionally adding regional labels or reference zones (``Northeast'',
``Southwest'', etc) could further support spatial variation.

\subsection{3.5 Event Analysis}\label{event-analysis}

\pandocbounded{\includegraphics[keepaspectratio]{Income_Inequality_files/figure-latex/unnamed-chunk-19-1.pdf}}

The temporal visualization shows that inequality rose significantly
after 1994, coinciding with the implementation of the North American
Free Trade Agreement (NAFTA). NAFTA, established between the U.S.,
Mexico and Canada, came into effect in January 1994, aiming to boost
trade and economic integration by removing trade barriers and import
tariffs (Floyd, 2025). As a result, exports and imports became cheaper,
and a significant share of U.S. manufacturing was relocated to Mexico
due to lower labor costs (Scott, 2011).

While NAFTA established the world's largest free trade zone, the
benefits were uneven. Higher-income groups gained the most, as corporate
profits increased and primarily flowed to shareholders and high-income
individuals (This trend explains the rise in the mean income, while the
median grew more slowly).

At the same time, many low-skilled workers were negatively affected. The
U.S. lost an estimated 682,900 manufacturing jobs by 2010, as
outsourcing intensified (Scott, 2011). As some workers faced stagnating
or declining incomes, those in high-skill sectors saw gains, widening
the income gap.

However, inequality cannot be contributed to NAFTA alone. Technological
change, labor market shifts, and globalization also contributed
significantly (IMF, 2017). Therefore, while the timing suggests a
potential link, the analysis does not confirm a causal relationship.

\section{4 Discussion}\label{discussion}

\subsection{4.1 Discussion of the
findings}\label{discussion-of-the-findings}

The analysis has shown that inequality in the US is structural growing
and a more urgent social issue. By measuring inequality through the top
10\% versus the bottom 90\% ratio, we find a general increase across
nearly all U.S states. The most significant increase occurred in
Massachusetts, California, Arizona, and Nebraska. These changes may be
linked to regional economic factors in specific areas. For example
California and Massachusetts had a rapid growth in tech and finance
industries during the 1990s, where highly skilled workers (often in the
top 10\%) were paid significantly more than lower income groups.

The sharp increase in inequality after 1994 coincides with the
implementation of NAFTA. NAFTA, as discussed earlier, led to job
outsourcing that likely benefited higher earners while disadvantaging
low-skilled workers.

The sub-population analysis, based on the share of people with a
bachelor's degree or higher, showed that states who have a higher
education level, often tend to have higher income inequality. Highly
educated workers have more access to better-paying jobs, which further
widens the income gap. These findings confirm previous research
indicating that populations with higher levels of education are more
unequal (Horowitz et al., 2020; Basu \& Stiglitz, 2016). This means
simply raising education levels may not reduce inequality. To close the
gap, other measures, such as fair wages and better job opportunities for
lower-educated workers, are also needed.

To conclude, our analysis has shown that income inequality is widespread
in the United States. It is most prominent in states with higher levels
of educational attainment and increased significantly following the
implementation of NAFTA.

\section{5 Reproducibility}\label{reproducibility}

\subsection{5.1 Github Repository Link}\label{github-repository-link}

\url{https://github.com/juliakoeleman/Income-Inequality-US.git}

\subsection{5.2 Reference List}\label{reference-list}

Arčabić, V., Kim, K. T., You, Y., \& Lee, J. (2021). Century-long
dynamics and convergence of income inequality among the US states.
Economic Modelling, 101, 105526.
\url{https://doi.org/10.1016/j.econmod.2021.105526}

Basu, K., \& Stiglitz, J. E. (2016). Inequality and Growth: Patterns and
Policy Volume II: Regions and Regularities. Palgrave Macmillan.
\url{https://doi.org/10.1057/9781137554598}

Denk, O., Hagemann, R. P., Lenain, P., Somma, V. (2013). Inequality and
Poverty in the United States: Public Policies for Inclusive Growth.
In~\emph{OECD Economics Department Working Papers} (No.~1052).
\url{https://dx.doi.org/10.1787/5k46957cwv8q-en}

Floyd, D. (2025). \emph{How Did NAFTA Affect the Economies of
Participating Countries?} Investopedia.
\url{https://www.investopedia.com/articles/economics/08/north-american-free-trade-agreement.asp\#:~:text=Cons\%20of\%20NAFTA\%20Many\%20jobs\%20shifted\%20from,suppress\%20wages\%20and\%20opportunities\%20for\%20several\%20parties}.

Horowitz, J., Igielnik, R., Kochhar, R. (2020). \emph{Most Americans Say
There Is Too Much Economic Inequality in the U.S., but Fewer Than Half
Call It a Top Priority}.
\url{https://www.pewsocialtrends.org/wp-content/uploads/sites/3/2020/01/PSDT_01.09.20_economic-inequailty_FULL.pdf}

International Monetary Fund. (2017). Fiscal monitor: tackling
inequality. In \emph{World Economic And Financial Surveys}.

Scott, R. E. (2011). Heading South: U.S.-Mexico trade and job
displacement after NAFTA (Briefing Paper \#308). Economic Policy
Institute.

Wei, Y. D. (2015). Spatiality of regional inequality. Applied Geography,
61, 1--10. \url{https://dx.doi.org/10.1016/j.apgeog.2015.03.013}

\end{document}
