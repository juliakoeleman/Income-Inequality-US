% Options for packages loaded elsewhere
\PassOptionsToPackage{unicode}{hyperref}
\PassOptionsToPackage{hyphens}{url}
%
\documentclass[
]{article}
\usepackage{amsmath,amssymb}
\usepackage{iftex}
\ifPDFTeX
  \usepackage[T1]{fontenc}
  \usepackage[utf8]{inputenc}
  \usepackage{textcomp} % provide euro and other symbols
\else % if luatex or xetex
  \usepackage{unicode-math} % this also loads fontspec
  \defaultfontfeatures{Scale=MatchLowercase}
  \defaultfontfeatures[\rmfamily]{Ligatures=TeX,Scale=1}
\fi
\usepackage{lmodern}
\ifPDFTeX\else
  % xetex/luatex font selection
\fi
% Use upquote if available, for straight quotes in verbatim environments
\IfFileExists{upquote.sty}{\usepackage{upquote}}{}
\IfFileExists{microtype.sty}{% use microtype if available
  \usepackage[]{microtype}
  \UseMicrotypeSet[protrusion]{basicmath} % disable protrusion for tt fonts
}{}
\makeatletter
\@ifundefined{KOMAClassName}{% if non-KOMA class
  \IfFileExists{parskip.sty}{%
    \usepackage{parskip}
  }{% else
    \setlength{\parindent}{0pt}
    \setlength{\parskip}{6pt plus 2pt minus 1pt}}
}{% if KOMA class
  \KOMAoptions{parskip=half}}
\makeatother
\usepackage{xcolor}
\usepackage[margin=1in]{geometry}
\usepackage{color}
\usepackage{fancyvrb}
\newcommand{\VerbBar}{|}
\newcommand{\VERB}{\Verb[commandchars=\\\{\}]}
\DefineVerbatimEnvironment{Highlighting}{Verbatim}{commandchars=\\\{\}}
% Add ',fontsize=\small' for more characters per line
\usepackage{framed}
\definecolor{shadecolor}{RGB}{248,248,248}
\newenvironment{Shaded}{\begin{snugshade}}{\end{snugshade}}
\newcommand{\AlertTok}[1]{\textcolor[rgb]{0.94,0.16,0.16}{#1}}
\newcommand{\AnnotationTok}[1]{\textcolor[rgb]{0.56,0.35,0.01}{\textbf{\textit{#1}}}}
\newcommand{\AttributeTok}[1]{\textcolor[rgb]{0.13,0.29,0.53}{#1}}
\newcommand{\BaseNTok}[1]{\textcolor[rgb]{0.00,0.00,0.81}{#1}}
\newcommand{\BuiltInTok}[1]{#1}
\newcommand{\CharTok}[1]{\textcolor[rgb]{0.31,0.60,0.02}{#1}}
\newcommand{\CommentTok}[1]{\textcolor[rgb]{0.56,0.35,0.01}{\textit{#1}}}
\newcommand{\CommentVarTok}[1]{\textcolor[rgb]{0.56,0.35,0.01}{\textbf{\textit{#1}}}}
\newcommand{\ConstantTok}[1]{\textcolor[rgb]{0.56,0.35,0.01}{#1}}
\newcommand{\ControlFlowTok}[1]{\textcolor[rgb]{0.13,0.29,0.53}{\textbf{#1}}}
\newcommand{\DataTypeTok}[1]{\textcolor[rgb]{0.13,0.29,0.53}{#1}}
\newcommand{\DecValTok}[1]{\textcolor[rgb]{0.00,0.00,0.81}{#1}}
\newcommand{\DocumentationTok}[1]{\textcolor[rgb]{0.56,0.35,0.01}{\textbf{\textit{#1}}}}
\newcommand{\ErrorTok}[1]{\textcolor[rgb]{0.64,0.00,0.00}{\textbf{#1}}}
\newcommand{\ExtensionTok}[1]{#1}
\newcommand{\FloatTok}[1]{\textcolor[rgb]{0.00,0.00,0.81}{#1}}
\newcommand{\FunctionTok}[1]{\textcolor[rgb]{0.13,0.29,0.53}{\textbf{#1}}}
\newcommand{\ImportTok}[1]{#1}
\newcommand{\InformationTok}[1]{\textcolor[rgb]{0.56,0.35,0.01}{\textbf{\textit{#1}}}}
\newcommand{\KeywordTok}[1]{\textcolor[rgb]{0.13,0.29,0.53}{\textbf{#1}}}
\newcommand{\NormalTok}[1]{#1}
\newcommand{\OperatorTok}[1]{\textcolor[rgb]{0.81,0.36,0.00}{\textbf{#1}}}
\newcommand{\OtherTok}[1]{\textcolor[rgb]{0.56,0.35,0.01}{#1}}
\newcommand{\PreprocessorTok}[1]{\textcolor[rgb]{0.56,0.35,0.01}{\textit{#1}}}
\newcommand{\RegionMarkerTok}[1]{#1}
\newcommand{\SpecialCharTok}[1]{\textcolor[rgb]{0.81,0.36,0.00}{\textbf{#1}}}
\newcommand{\SpecialStringTok}[1]{\textcolor[rgb]{0.31,0.60,0.02}{#1}}
\newcommand{\StringTok}[1]{\textcolor[rgb]{0.31,0.60,0.02}{#1}}
\newcommand{\VariableTok}[1]{\textcolor[rgb]{0.00,0.00,0.00}{#1}}
\newcommand{\VerbatimStringTok}[1]{\textcolor[rgb]{0.31,0.60,0.02}{#1}}
\newcommand{\WarningTok}[1]{\textcolor[rgb]{0.56,0.35,0.01}{\textbf{\textit{#1}}}}
\usepackage{graphicx}
\makeatletter
\newsavebox\pandoc@box
\newcommand*\pandocbounded[1]{% scales image to fit in text height/width
  \sbox\pandoc@box{#1}%
  \Gscale@div\@tempa{\textheight}{\dimexpr\ht\pandoc@box+\dp\pandoc@box\relax}%
  \Gscale@div\@tempb{\linewidth}{\wd\pandoc@box}%
  \ifdim\@tempb\p@<\@tempa\p@\let\@tempa\@tempb\fi% select the smaller of both
  \ifdim\@tempa\p@<\p@\scalebox{\@tempa}{\usebox\pandoc@box}%
  \else\usebox{\pandoc@box}%
  \fi%
}
% Set default figure placement to htbp
\def\fps@figure{htbp}
\makeatother
\setlength{\emergencystretch}{3em} % prevent overfull lines
\providecommand{\tightlist}{%
  \setlength{\itemsep}{0pt}\setlength{\parskip}{0pt}}
\setcounter{secnumdepth}{-\maxdimen} % remove section numbering
\usepackage{bookmark}
\IfFileExists{xurl.sty}{\usepackage{xurl}}{} % add URL line breaks if available
\urlstyle{same}
\hypersetup{
  pdftitle={The Effect of Education Level on Income Inequality},
  pdfauthor={Carlijn Calori, Leah Delikát, Fadhil Dhafir, Sten Groen, Julia Koeleman, Anne Schrama, Marie-Louise Stevens, Sophia Zentgraf},
  hidelinks,
  pdfcreator={LaTeX via pandoc}}

\title{The Effect of Education Level on Income Inequality}
\author{Carlijn Calori, Leah Delikát, Fadhil Dhafir, Sten Groen, Julia
Koeleman, Anne Schrama, Marie-Louise Stevens, Sophia Zentgraf}
\date{2025-06-24}

\begin{document}
\maketitle

\#Set up your environment

\#Title Page Carlijn Calori, Leah Delikát, Fadhil Dhafir, Sten Groen,
Julia Koeleman, Anne Schrama, Marie-Louise Stevens, Sophia Zentgraf

Tutorial: Group 1

J.F. Fitzgerald

\section{1 Identification of the Social
Problem}\label{identification-of-the-social-problem}

\#\#1.1 Describe the Social Problem Income inequality is a serious
social issue in the United States. Over the past few decades, the gap
between the highest and lowest earners has grown a lot, leading to
problems on social, economic, and political levels. This widening gap
creates unequal opportunities, adds pressure to communities, and makes
it harder for people to succeed in the job market. Education plays an
important role in this situation. People with higher levels of education
tend to earn more money than those with lower levels of education. But
how straightforward is that relationship?

Several well-known sources have pointed out that income inequality is a
major concern. The Organisation for Economic Co-operation and
Development (OECD) has stated in multiple reports that the U.S. has one
of the highest levels of income inequality among developed countries.
They also point out that this can slow down economic growth and limit
social mobility. The Pew Research Center has also found that income gaps
keep growing and that education is a key factor---people without higher
education are falling further behind in the labor market. These sources
show that income inequality, and the role education plays in it, is a
long-term issue that needs real attention.

Even though this topic has been researched before, our project takes a
closer look than many past studies. We focus on differences between U.S.
states to see how the link between education and income changes from one
region to another. By doing this, we can offer a more detailed and
complete picture of income inequality. With this more in-depth analysis,
we hope to find new insights that haven't been explored in other
research.

\url{https://www.pewresearch.org/social-trends/2020/01/09/trends-in-income-and-wealth-inequality/}?
\url{https://www.oecd.org/en/publications/inequality-and-poverty-in-the-united-states_5k46957cwv8q-en.html}

\section{2 Data Sourcing}\label{data-sourcing}

\subsection{2.1 Description and
limitations}\label{description-and-limitations}

The dataset ``Inequality and Growth in the United States: Evidence from
a New State-Level Panel of Income Inequality Measure'' of the researcher
Frank has been used for our project. This dataset provides historical
coverage (1990-2000) of the highest income group in the US. Another
reason why this data set has been used is because it specifically shows
the top income groups per state. This way, the differences per state can
be easily compared to one another. The dataset shows multiple variables
that can measure inequality. The data is not from a government website,
but from a researcher himself. However, the researcher did use data from
the World Inequality Database to calculate the different top income
groups. The limitation of this dataset is that it only provides the top
income percentages, so there is not a possibility to compare the bottom
10 percent with the top 10, which makes it harder to accurately measure
inequality.

The second dataset is ``The Educational attainment of persons 25 years
old and over, by race/ethnicity and state: April 1990 and April 2000''.
This dataset focuses on education level and has been retrieved from the
National Center form Education Statistics (NCES). We chose this dataset
because the level of education is a key indicator of inequality. The
higher the education, the higher the income. Therefore big differences
between education level will lead to a greater inequality. Furthermore,
the dataset specifies the education level across all U.S. states. Which
we can combine with the dataset of the top income percentages.
Differences across racial and ethnic groups are also shown in the data,
but will not be considered here, as they fall outside the scope of this
analysis. The limitation of the dataset is that it includes data about
only two points in time: April 1990 and April 2000. Unfortunately, there
is no data available for the years in between.

\subsection{2.2 Load in the data}\label{load-in-the-data}

\begin{Shaded}
\begin{Highlighting}[]
\CommentTok{\#Load the files}
\NormalTok{Distribution }\OtherTok{\textless{}{-}} \FunctionTok{read\_excel}\NormalTok{(}\StringTok{"datasets\_income\_inequality/Frank\_WID\_2020.xls"}\NormalTok{, }\AttributeTok{sheet =} \DecValTok{3}\NormalTok{)}
\NormalTok{Education }\OtherTok{\textless{}{-}} \FunctionTok{read\_excel}\NormalTok{(}\StringTok{"datasets\_income\_inequality/tabn012.xls"}\NormalTok{, }\AttributeTok{col\_types =} \FunctionTok{c}\NormalTok{(}\StringTok{"text"}\NormalTok{, }\StringTok{"skip"}\NormalTok{, }\StringTok{"text"}\NormalTok{, }\StringTok{"skip"}\NormalTok{, }\StringTok{"skip"}\NormalTok{, }\StringTok{"skip"}\NormalTok{, }\StringTok{"skip"}\NormalTok{, }\StringTok{"skip"}\NormalTok{, }\StringTok{"skip"}\NormalTok{, }\StringTok{"skip"}\NormalTok{, }\StringTok{"skip"}\NormalTok{, }\StringTok{"skip"}\NormalTok{, }\StringTok{"skip"}\NormalTok{, }\StringTok{"skip"}\NormalTok{, }\StringTok{"text"}\NormalTok{, }\StringTok{"skip"}\NormalTok{, }\StringTok{"skip"}\NormalTok{, }\StringTok{"skip"}\NormalTok{, }\StringTok{"skip"}\NormalTok{, }\StringTok{"skip"}\NormalTok{, }\StringTok{"skip"}\NormalTok{, }\StringTok{"skip"}\NormalTok{, }\StringTok{"skip"}\NormalTok{, }\StringTok{"skip"}\NormalTok{, }\StringTok{"skip"}\NormalTok{, }\StringTok{"skip"}\NormalTok{, }\StringTok{"text"}\NormalTok{, }\StringTok{"skip"}\NormalTok{, }\StringTok{"skip"}\NormalTok{, }\StringTok{"skip"}\NormalTok{, }\StringTok{"skip"}\NormalTok{, }\StringTok{"skip"}\NormalTok{, }\StringTok{"skip"}\NormalTok{,}\StringTok{"skip"}\NormalTok{, }\StringTok{"skip"}\NormalTok{, }\StringTok{"skip"}\NormalTok{, }\StringTok{"skip"}\NormalTok{, }\StringTok{"skip"}\NormalTok{, }\StringTok{"text"}\NormalTok{, }\StringTok{"skip"}\NormalTok{, }\StringTok{"skip"}\NormalTok{, }\StringTok{"skip"}\NormalTok{, }\StringTok{"skip"}\NormalTok{, }\StringTok{"skip"}\NormalTok{, }\StringTok{"skip"}\NormalTok{, }\StringTok{"skip"}\NormalTok{, }\StringTok{"skip"}\NormalTok{, }\StringTok{"skip"}\NormalTok{, }\StringTok{"skip"}\NormalTok{))}
\end{Highlighting}
\end{Shaded}

\begin{verbatim}
## New names:
## * `` -> `...2`
## * `` -> `...3`
## * `` -> `...4`
## * `` -> `...5`
\end{verbatim}

\subsection{3.1 Summary of the datasets}\label{summary-of-the-datasets}

\subsection{3.2 Describing the type of variables in the
datasets}\label{describing-the-type-of-variables-in-the-datasets}

This is our merged data set. In the first column you can see what year
the data refers to. Next to it is the column with the State\_ID. We put
the states in alphabetical order and then started numbering from 1 to
100 (example: Alabama is numbered 1 in 1990 and 51 (1+50) in 2000).

The next 5 columns show the percentages of total income that a group
earns. For example, the first column shows that the top 10 percent of
big earners, earn 39.9\% of total income, from the state. This is the
same for the 4 other columns, but here the groups get smaller and
smaller.

Columns 9 through 14 show how much you have to earn to belong to that
percentage of big earners. For example, in 1990 you had to earn more
than \$101,762.24 to be among the top 10 big earners in Alabama. Did you
want to be among the top 5 in 1990? Then you had to earn more than
\$131,506.30.

The last two columns show level of education. These variables are in
percent. So for example, in 1990, 66.9\% of the population in Alabama
had a high school diploma or higher and 15.7\% had a bachelor's degree
or higher.

\section{3 Quantifying}\label{quantifying}

\subsection{3.1 Data Cleaning}\label{data-cleaning}

Only the years 1990 to 2000 are kept in the Distribution dataset. This
is because the Education dataset includes data only for 1990 and 2000.
To ensure the datasets can be merged and compared properly, just 1990,
2000, and the years in between are included.

In addition, two rows are removed, as they are not a U.S. state. One of
the rows corresponds to the entire United States, while the other
represents the District of Columbia.

\begin{Shaded}
\begin{Highlighting}[]
\NormalTok{Distribution }\OtherTok{\textless{}{-}}\NormalTok{ Distribution }\SpecialCharTok{\%\textgreater{}\%}
  \FunctionTok{filter}\NormalTok{(Year }\SpecialCharTok{\%in\%} \DecValTok{1990}\SpecialCharTok{:}\DecValTok{2000}\NormalTok{, }
\NormalTok{         State }\SpecialCharTok{!=} \FunctionTok{c}\NormalTok{(}\StringTok{"United States"}\NormalTok{, }\StringTok{"District of Columbia"}\NormalTok{))}
\end{Highlighting}
\end{Shaded}

For the Education dataset, unnecessary rows are removed. These rows
contain no data and consist only of blank spaces between entries. Also,
two rows representing the entire United States and the District of
Columbia are removed.

To simplify the datatset, columns are renamed to reflect the type of
data they contain, and any dots following state names are removed.

For the final step of cleaning, before merging the two datasets, the
Education dataset needs to be converted to a long format. Currently, it
contains four separate variables for the years 1990 and 2000, but these
need to be combined into a single column with the years listed
vertically. This transformation allows the dataset to be merged properly
with the Distribution dataset, which is already in long format.

\begin{Shaded}
\begin{Highlighting}[]
\NormalTok{Education }\OtherTok{\textless{}{-}}\NormalTok{ Education[ }\SpecialCharTok{{-}}\FunctionTok{c}\NormalTok{(}\DecValTok{1}\SpecialCharTok{:}\DecValTok{13}\NormalTok{, }\DecValTok{14}\NormalTok{, }\DecValTok{20}\NormalTok{, }\DecValTok{24}\NormalTok{, }\DecValTok{26}\NormalTok{, }\DecValTok{32}\NormalTok{, }\DecValTok{38}\NormalTok{, }\DecValTok{44}\NormalTok{, }\DecValTok{50}\NormalTok{, }\DecValTok{56}\NormalTok{, }\DecValTok{62}\NormalTok{, }\DecValTok{68}\NormalTok{, }\DecValTok{75}\SpecialCharTok{:}\DecValTok{79}\NormalTok{), ]}
\end{Highlighting}
\end{Shaded}

\begin{Shaded}
\begin{Highlighting}[]
\FunctionTok{names}\NormalTok{(Education)[}\DecValTok{1}\SpecialCharTok{:}\DecValTok{5}\NormalTok{] }\OtherTok{\textless{}{-}} \FunctionTok{c}\NormalTok{(}\StringTok{"State"}\NormalTok{, }\StringTok{"Highschool1990"}\NormalTok{, }\StringTok{"Highschool2000"}\NormalTok{, }\StringTok{"Bachelor1990"}\NormalTok{, }\StringTok{"Bachelor2000"}\NormalTok{)}
\end{Highlighting}
\end{Shaded}

\begin{Shaded}
\begin{Highlighting}[]
\NormalTok{Education }\OtherTok{\textless{}{-}}\NormalTok{ Education }\SpecialCharTok{\%\textgreater{}\%} 
  \FunctionTok{mutate}\NormalTok{(}\AttributeTok{State =}\NormalTok{ State }\SpecialCharTok{\%\textgreater{}\%}
           \FunctionTok{str\_replace\_all}\NormalTok{(}\StringTok{"…+"}\NormalTok{, }\StringTok{""}\NormalTok{) }\SpecialCharTok{\%\textgreater{}\%}
           \FunctionTok{str\_replace\_all}\NormalTok{(}\StringTok{"}\SpecialCharTok{\textbackslash{}\textbackslash{}}\StringTok{.+"}\NormalTok{, }\StringTok{""}\NormalTok{) }\SpecialCharTok{\%\textgreater{}\%}
           \FunctionTok{str\_trim}\NormalTok{()) }\CommentTok{\# to delete all the dots after the state }
\end{Highlighting}
\end{Shaded}

\begin{Shaded}
\begin{Highlighting}[]
\NormalTok{df1990 }\OtherTok{\textless{}{-}} \FunctionTok{data.frame}\NormalTok{(}\AttributeTok{State =}\NormalTok{ Education}\SpecialCharTok{$}\NormalTok{State, }\AttributeTok{Year =} \DecValTok{1990}\NormalTok{, }\AttributeTok{Highschool =}\NormalTok{ Education}\SpecialCharTok{$}\NormalTok{Highschool1990, }\AttributeTok{Bachelor =}\NormalTok{ Education}\SpecialCharTok{$}\NormalTok{Bachelor1990, }\AttributeTok{stringsAsFactors =} \ConstantTok{FALSE}\NormalTok{)}

\NormalTok{df2000 }\OtherTok{\textless{}{-}} \FunctionTok{data.frame}\NormalTok{(}\AttributeTok{State =}\NormalTok{ Education}\SpecialCharTok{$}\NormalTok{State,}\AttributeTok{Year =} \DecValTok{2000}\NormalTok{,}\AttributeTok{Highschool =}\NormalTok{ Education}\SpecialCharTok{$}\NormalTok{Highschool2000,}\AttributeTok{Bachelor =}\NormalTok{ Education}\SpecialCharTok{$}\NormalTok{Bachelor2000,}\AttributeTok{stringsAsFactors =} \ConstantTok{FALSE}\NormalTok{)}

\NormalTok{Education }\OtherTok{\textless{}{-}} \FunctionTok{rbind}\NormalTok{(df1990, df2000)}

\NormalTok{Education }\OtherTok{\textless{}{-}}\NormalTok{ Education }\SpecialCharTok{\%\textgreater{}\%}
  \FunctionTok{mutate}\NormalTok{(}\FunctionTok{across}\NormalTok{(}\DecValTok{3}\SpecialCharTok{:}\DecValTok{4}\NormalTok{, }\SpecialCharTok{\textasciitilde{}} \FunctionTok{as.numeric}\NormalTok{(.)))}
\end{Highlighting}
\end{Shaded}

To merge the datasets, the full\_join() function is applied. This
function is appropriate because it performs an outer join, retaining all
observations from both datasets based on the specific join keys (e.g.,
State and Year). Since the distribution dataset includes data for each
year from 1990 to 2000, and the Education dataset only includes data for
1990 and 2000, full\_join() ensures that no rows are lost during the
merge. Missing values are introduced where data is not available in one
of the datasets, allowing for a complete and allinged structure for
subsequent analysis. After the merge, all numeric values are rounded to
two decimals.

\begin{Shaded}
\begin{Highlighting}[]
\NormalTok{Inequality }\OtherTok{\textless{}{-}}\NormalTok{ Distribution }\SpecialCharTok{\%\textgreater{}\%}
  \FunctionTok{full\_join}\NormalTok{(Education, }\AttributeTok{by =} \FunctionTok{c}\NormalTok{(}\StringTok{"State"}\NormalTok{, }\StringTok{"Year"}\NormalTok{))}

\NormalTok{Inequality }\OtherTok{\textless{}{-}}\NormalTok{ Inequality }\SpecialCharTok{\%\textgreater{}\%}
  \FunctionTok{select}\NormalTok{(}\SpecialCharTok{{-}}\NormalTok{st) }\SpecialCharTok{\%\textgreater{}\%}
  \FunctionTok{mutate}\NormalTok{(}\FunctionTok{across}\NormalTok{(}\FunctionTok{where}\NormalTok{(is.numeric), }\SpecialCharTok{\textasciitilde{}} \FunctionTok{round}\NormalTok{(.x, }\DecValTok{2}\NormalTok{)))}
\end{Highlighting}
\end{Shaded}

\subsection{3.2 Creating new variables}\label{creating-new-variables}

To measure income inequality, a new variable is constructed, modeled
after the Palma Ratio. This ratio is defined as ``the ratio of the
income share of the top 10 percent over that of the bottom 40 percent''
(Basu \& Stiglitz, 2016, p.~xxvi), and is designed to capture inequality
at the extremes of the income distribution. However, the Distribution
dataset does not include the income share of the bottom 40 percent.
Therefore, as a practical and intuitive alternative, inequality is
measured by dividing the income share of the top 10 percent by that of
the bottom 90 percent (i.e., 100\% minus the top 10\%).

Using this measure, inequality will be visualized over time by
presenting the mean and median values of all U.S. states. This
visualization provides a broad overview of the national trend in
inequality across the country, showing whether inequality is generally
increasing or decreasing over time.

Additionally, a second new variable groups states into four categories -
Low, Medium Low, Medium High, and High - based on the percentage of the
population with a bachelor's degree or higher, allowing inequality to be
examined across sub-populations. This approach helps reveal whether
states with a higher share of highly educated residents tend to have a
higher or lower inequality.

Finally, inequality will be visualized across the 50 states through a
third new variable that shows the absolute change in inequality between
1990 and 2000. Not all states experience inequality changes equally;
some show significant increases while others remain stable. This
variation can be attributed to factors such as local institutions, tax
policies, or demographic shifts (Wei, 2015; Arcabic et al., 2021).

\begin{Shaded}
\begin{Highlighting}[]
\NormalTok{Inequality }\OtherTok{\textless{}{-}}\NormalTok{ Inequality }\SpecialCharTok{\%\textgreater{}\%}
  \FunctionTok{mutate}\NormalTok{(}\AttributeTok{Top10\_vs\_bottom90 =} \FunctionTok{round}\NormalTok{(Top10\_adj}\SpecialCharTok{/}\NormalTok{(}\DecValTok{100}\SpecialCharTok{{-}}\NormalTok{Top10\_adj), }\DecValTok{2}\NormalTok{))}
\end{Highlighting}
\end{Shaded}

\begin{Shaded}
\begin{Highlighting}[]
\NormalTok{Inequality }\OtherTok{\textless{}{-}}\NormalTok{ Inequality }\SpecialCharTok{\%\textgreater{}\%}
  \FunctionTok{group\_by}\NormalTok{(Year) }\SpecialCharTok{\%\textgreater{}\%}
  \FunctionTok{mutate}\NormalTok{(}\AttributeTok{Quartile =} \FunctionTok{ntile}\NormalTok{(Bachelor, }\DecValTok{4}\NormalTok{), }\AttributeTok{Bachelor\_level =} \FunctionTok{case\_when}\NormalTok{(Quartile }\SpecialCharTok{==} \DecValTok{1} \SpecialCharTok{\textasciitilde{}} \StringTok{"Low"}\NormalTok{, Quartile }\SpecialCharTok{==} \DecValTok{2} \SpecialCharTok{\textasciitilde{}} \StringTok{"Medium Low"}\NormalTok{, Quartile }\SpecialCharTok{==} \DecValTok{3} \SpecialCharTok{\textasciitilde{}} \StringTok{"Medium High"}\NormalTok{,Quartile }\SpecialCharTok{==} \DecValTok{4} \SpecialCharTok{\textasciitilde{}} \StringTok{"High"}\NormalTok{))}

\CommentTok{\# Make sure Education is a factor with the right order}
\NormalTok{Inequality}\SpecialCharTok{$}\NormalTok{Bachelor\_level }\OtherTok{\textless{}{-}} \FunctionTok{factor}\NormalTok{(}
\NormalTok{  Inequality}\SpecialCharTok{$}\NormalTok{Bachelor\_level,}
  \AttributeTok{levels =} \FunctionTok{c}\NormalTok{(}\StringTok{"Low"}\NormalTok{, }\StringTok{"Medium Low"}\NormalTok{, }\StringTok{"Medium High"}\NormalTok{, }\StringTok{"High"}\NormalTok{))}
\end{Highlighting}
\end{Shaded}

\subsection{3.3 Temporal Variation}\label{temporal-variation}

\pandocbounded{\includegraphics[keepaspectratio]{Income_Inequality_files/figure-latex/unnamed-chunk-13-1.pdf}}

\subsection{3.4 Sub-population
Variation}\label{sub-population-variation}

\pandocbounded{\includegraphics[keepaspectratio]{Income_Inequality_files/figure-latex/unnamed-chunk-14-1.pdf}}

\subsection{3.5 Spatial Variation}\label{spatial-variation}

\begin{verbatim}
## Warning: The `legend.text.align` argument of `theme()` is deprecated as of ggplot2
## 3.5.0.
## i Please use theme(legend.text = element_text(hjust)) instead.
## This warning is displayed once every 8 hours.
## Call `lifecycle::last_lifecycle_warnings()` to see where this warning was
## generated.
\end{verbatim}

\pandocbounded{\includegraphics[keepaspectratio]{Income_Inequality_files/figure-latex/unnamed-chunk-17-1.pdf}}

\subsection{3.5 Event Analysis}\label{event-analysis}

\begin{verbatim}
## Warning: Using `size` aesthetic for lines was deprecated in ggplot2 3.4.0.
## i Please use `linewidth` instead.
## This warning is displayed once every 8 hours.
## Call `lifecycle::last_lifecycle_warnings()` to see where this warning was
## generated.
\end{verbatim}

\pandocbounded{\includegraphics[keepaspectratio]{Income_Inequality_files/figure-latex/unnamed-chunk-18-1.pdf}}

The temporal visualization shows that inequality rose significantly
after 1994, which may be linked to the implementation of the North
American Free Trade Agreement (NAFTA). NAFTA, established between the
U.S., MExico and Canada came into effect in January 1994 with the
objective of creating a trilateral trade bloc to increase economic
integration, boost trade, and stimulate growth by removing trade
barriers and import tariffs. As a result, exports and imports became
cheaper, and a significant share of U.S. manufacturing was relocated to
Mexico due to lower labor costs (SOURCE).

NAFTA established the world's largest free trade zone, but not everyone
benefited equally. Higher-income groups in the U.S. gained the most, as
large corporations experienced increased profits that mainly flowed to
shareholders and high-income individuals (This trend explains the rise
in the mean income, while the median grew more slowly)(SOURCE).

While the NAFTA boosted overall GDP, it disproportionately affected
lower-income groups. The United States lost numerous manufacturing jobs
(Estimated at 682,900 by 2010) as companies relocated production to
Mexico to cut labor costs (Scott, 2011). This acceleration of
outsourcing significantly impacted American manufacturing and
low-skilled workers. Consequently, some workers saw their income
stagnate or decline, whereas individuals in management and high-skill
sectors experienced income growth (SOURCE). As a result, income
inequality widened, further expanding the gap between the bottom 90
percent and the top earners.

However, it is important to recognize that the rise in income inequality
cannot be attributed solely to NAFTA. Factors such as technological
advancements, shifts in labor market policies and broader globalization
have also played significant roles in driving this trend.

\section{4 Discussion}\label{discussion}

\subsection{4.1 Discussion of the
findings}\label{discussion-of-the-findings}

This project has shown that inequality in the US is structural growing
and a more urgent social issue. The income share ratio that we
constructed shows that there is an increase in inequality between 1990
and 2000. This increase is visible in almost every US state. The largest
increases occurred in Massachusetts, California, Arizona, and Nebraska.
The changes in inequality may be linked to regional economic factors in
specific areas. For example California and Massachusetts had a rapid
growth in tech and finance industries during the 1990s, where highly
skilled workers (often in the top 10\%) were paid significantly more
than lower income groups. The analysis that can be made from our dataset
is that the income inequality started rising more starting from the year
1994. This year, the Nort American Free Trade Agreement (NAFTA) has been
implemented. The goal of NAFTA was to stimulate trade and economic
growth, but it this agreement also led to outsourcing to lower-wage
countries. Because of this outsourcing, many low-skilled manufacturing
job in the US were lost. While the lower class was struggling, high
income groups were benefiting from rising corporate profits and stock
value. Which is why the inequality started to rise significantly more.
The subpopulation, based on the share of people with a bachelor's degree
or higher, showed that states who have a higher education level, often
tend to have a greater inequality. The connection between this relation
is that highly educated workers have more access to better-paying jobs,
which further widens the income gap. Our analysis shows that inequality
is widespread but most prominent in states with higher education levels
and following policy shifts such as NAFTA.

\section{5 Reproductibility}\label{reproductibility}

\subsection{5.1 Github Repository link}\label{github-repository-link}

\subsection{5.2 Reference list}\label{reference-list}

Pew Research Center. (2020, January 9). Trends in U.S. income and wealth
inequality. Pew Research Center. Retrieved from
\url{https://www.pewresearch.org/social-trends/2020/01/09/trends-in-income-and-wealth-inequality/}

Denk, O., Hagemann, R., Lenain, P., \& Somma, V. (2013, May 27).
Inequality and poverty in the United States: Public policies for
inclusive growth (OECD Working Paper No.~1052). OECD. Retrieved from
\url{https://www.oecd.org/en/publications/inequality-and-poverty-in-the-united-states_5k46957cwv8q-en.html}

Scott, R. E. (2011, 3 mei). Heading South: U.S.-Mexico trade and job
displacement after NAFTA (Briefing Paper \#308). Economic Policy
Institute.
\url{http://www.epi.org/publications/entry/briefing_paper_308/}

\end{document}
